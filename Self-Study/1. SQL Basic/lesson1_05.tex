REFINING SELECTTION | STRONG WEAPON IN THE ARSENAL

KEYWORD: DISTINCT, ORDER BY, LIMIT, LIKE

===================================================================================
1. We can write code to add new books as follows:
INSERT INTO books
    (title, author_fname, author_lname, released_year, stock_quantity, pages)
    VALUES ('10% Happier', 'Dan', 'Harris', 2014, 29, 256), 
           ('fake_book', 'Freida', 'Harris', 2001, 287, 428),
           ('Lincoln In The Bardo', 'George', 'Saunders', 2017, 1000, 367);

===================================================================================
2. In case of selecting different values, we can write code as follow:
SELECT DISTINCT <feature_name> ....
The DISTINCT keyword select the entire different combination of columns in the SELECT clause, not just to a single column or feature.
For example:
    SELECT DISTINCT author_lname FROM books;
    SELECT DISTINCT CONCAT(author_fname,' ', author_lname) FROM books;
    SELECT DISTINCT author_fname, author_lname FROM books;

    
===================================================================================
3. For softing values (accourdnig to a specific feature), use "ORDER BY" syntax:
For example: 
    SELECT <feature_name> FROM <relation_name>
    ORDER BY <softed-feature> <DESC/ASC>;
with DESC for decrease value, ASC for increasing value;

If user lists a sequence of feature names, its ordered-number can be used for specifying the softed feature.
For example:
    SELECT book_id, author_fname, author_lname, pages
    FROM books
    ORDER BY 2 DESC;

Example:
    SELECT book_id, author_fname, author_lname, pages
    FROM books 
    ORDER BY author_lname, author_fname;
This query retrieves the specified columns from the books table and sorts the results alphabetically by author_lname.
Within each author_lname, if multiple authors share the same last name, it will then sort by author_fname.


===================================================================================
4. When you just want to retrieve first n queries, use "LIMIT n".
For example: 
    SELECT * FROM books LIMIT 1;
===================================================================================
5. In case of forgeting a/some character(s), use LIKE, mix and match with \% and _
%: Zero, one, or many characters
_: a character only.
For example:
    SELECT * FROM books
    WHERE author_fname LIKE '____';
Example 02:
    SELECT title, author_fname, author_lname, pages 
    FROM books
    WHERE author_fname LIKE '%da%';

-- To select books with '%' in their title:
    SELECT * FROM books
    WHERE title LIKE '%\%%';
 
-- To select books with an underscore '_' in title:
    SELECT * FROM books
    WHERE title LIKE '%\_%';

====================================================================================
